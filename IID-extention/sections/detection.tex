\section{Static Detection of IID Issues} \label{sec:detection}

We proposed a static IID issue detector, TAPIR, based on a set of anti-pattern rules extracted from our empirical study results.
This section describes the design (in Section~\ref{subsec:tapir-design}) and implementation (in Section~\ref{subsec:tapir-impl}) of TAPIR.

\subsection{IID Issue Anti-pattern Rules} \label{subsec:tapir-design}

By further inspecting the empirical study results and IID issue cases,
we observed that most IID issues are correlated with image decoding APIs concerning \emph{external} images,
which are essentially a small portion of all image decoding APIs.

In particular, only the nine following Android~\cite{Android_API} official APIs are correlated with IID%
\footnote{\code{setImageURI} and \code{setImageViewUri} both decode and display an image.}:

\begin{Verbatim}[fontsize=\footnotesize]
decodeFile      decodeFileDescriptor decodeStream
decodeByteArray setImageURI          decodeRegion
createFromPath  createFromStream     setImageViewUri
\end{Verbatim}

We also observed two popular third-party APIs (APIs invoking third-party library functionalities, not APIs used inside third-party libraries),
which are associated with at least two apps in the studied IID issues:

\begin{Verbatim}[fontsize=\footnotesize]
universalimageloader.core.ImageLoader.
  getInstance().displayImage
Glide.load
\end{Verbatim}
We call the above eleven image decoding and third-party APIs \emph{issue inducing APIs}. IID issues can occur when these APIs are invoked under issue-inducing \emph{rules},
which consist of API invocation sequence and/or parameter value combinations. These issue-inducing \emph{rules} are characterized in Table~\ref{tab:rules},
which are matched against in the TAPIR static analyzer.

\subsection{The TAPIR Static Analyzer} \label{subsec:tapir-impl}

We implemented the pattern-and-rule based static analyzer on top of Soot~\cite{soot}.
TAPIR takes an Android app binary (\code{apk}) file as input and uses \code{dex2jar}~\cite{dex2jar} to obtain a set of Java bytecode files.
It then builds the app's context-insensitive call graph,
with a few implicit method invocation relations being added, which are used to check rule \#4:

\begin{enumerate}
  \item The methods of \code{AsyncTask.execute} and \code{Async\allowbreak{}Task.\allowbreak{}doInBackground} in the same class have an implicit invocation relationship.
  \item The methods of \code{Thread.start} and \code{Thread.run} in the same class should have an implicit invocation relationship.
\end{enumerate}

%\newcommand{\isite}{i-site\xspace}

Then TAPIR checks each potential issue-inducing API invocation site (\emph{IS} for short) against the anti-pattern rules in Table~\ref{tab:rules} using standard program analysis techniques.
For each \emph{IS}, we can thus obtain:
(1) the data-flow of method parameters by a backward slicing, and
(2) the usages of decoded image objects by a forward slicing. In particular, TAPIR checks the anti-pattern rules as follows:

\begin{enumerate}
  \item Rule \#1 (\emph{image decoding without resizing}) is checked by analyzing the data-flow of the \code{Option} parameter,
  and a warning is raised if there lacks the \code{Option} parameter or its value satisfies the condition specified in Table~\ref{tab:rules}.
  \item Rule \#2 and \#3 (\emph{loop-based redundant image decoding}) are equivalent to checking the call graph reachability between the loop-related method invocations and the \emph{IS}. Furthermore, TAPIR also checks whether there is any data flow between the decoded image and cache-related functions or argument (in particular, \code{LruCache.put}, \code{DiskCacheStrategy.All} ) to exclude non-IID cases.
  \item Rule \#4 (\emph{image decoding in UI event handlers}) is another case of checking the reachability between the \emph{IS} and method invocations of \code{Thread.\allowbreak{}run}, \code{AsyncTask.\allowbreak{}Do\allowbreak{}In\allowbreak{}Background}, or \code{Intent\allowbreak{}Service.\allowbreak{}onHandleIntent}.
  \item Rules \#5, \#6, and \#7 (\emph{unbounded image caching}) follow the same pattern of checking whether a series of designated method invocations are reachable in the call graph.
\end{enumerate}

For each \emph{IS} matching at least one anti-pattern rule,
an inefficient image display warning is generated,
which can be further validated by the respective app developer.